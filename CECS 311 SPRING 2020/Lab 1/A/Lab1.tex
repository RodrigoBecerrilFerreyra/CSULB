\documentclass{article}

\usepackage{ragged2e}
\usepackage{graphicx}
\usepackage{amsmath}
\usepackage{siunitx}

\begin{document}

\begin{flushright}
    \noindent
    Rodrigo Becerril Ferreyra\\
    CECS 311 Section 01\\
    Lab 1\\
    2020-01-23 to 2020-01-28
\end{flushright}

\section{Introduction} In this lab, we were tasked with experimenting
with the idea of a diode---specifically a light-emitting
diode, or LED---and seeing how applying different voltages
to it affected how much current ran through the diode, and
how bright the LED lit up. A diode's main purpose is to
only allow current to run in one direction; LEDs also
happen to emit light while current runs through them. All
diodes have a forward voltage (\(V_f\)), which is the voltage
that must be applied to it in order for it to act as a wire
with very low resistance. Our LED also had a maximum current
(\(I_f\)), which is the maximum amount of current that
can safely go through it.

\section{Lab Setup} Our red LED came with a data sheet:
on this data sheet, it noted the forward voltage \(V_f\) as
between \SI{1.8}{\volt} and \SI{2.2}{V}. It also listed the
maximum current \(I_f\) as \SI{30}{mA}. Our circuit consisted
of the LED connected in series with a \SI{330}{\ohm} resistor.
We were tasked with applying voltages from \SI{0}{V}--\SI{5}{V}
in increments of \SI{0.25}{V}, and
taking measurements of
the current going through the circuit (\(I\)),
the voltage across the LED (\(V_D\)),
and the voltage across the resistor (\(V_R\)),
along with a short description of the
amount of light that the LED is emitting. The results are
listed below.

\begin{equation*}
\begin{array}{l | l | l | l | l}
        
    V_\text{applied} \text{ (\si{V})} & I \text{ (\si{mA})} & V_D\text{ (\si{V})} & V_R \text{ (\si{V})} & \text{Comments}\\ \hline
    0.50 & 1.6 & 0.018& 0.500& \text{Off}\\
    0.75 & 2.1 & 0.024& 0.682& \text{Off}\\
    1.00 & 3.0 & 0.035& 1.00 & \text{Off}\\
    1.25 & 3.8 & 0.043& 1.24 & \text{Off}\\
    1.50 & 4.5 & 0.13 & 1.49 & \text{Off}\\
    1.75 & 5.1 & 0.06 & 1.68 & \text{Slight glow}\\
    2.00 & 5.9 & 0.067& 1.94 & \text{Bright glow}\\
    2.25 & 6.3 & 0.071& 2.06 & \text{Very bright glow}\\
    2.50 & 7.2 & 0.078& 2.37 & \text{Very bright glow}\\
    2.75 & 7.6 & 0.086& 2.50 & \text{Very bright glow}\\
    3.00 & 8.8 & 0.10 & 2.92 & \text{Very bright glow}\\
    3.50 &10.5 & 0.12 & 3.47 & \text{Intense glow}\\
    4.00 &11.6 & 0.13 & 3.82 & \text{Max glow}\\
    4.50 &13.0 & 0.14 & 4.30 & \text{Max glow}\\
    5.00 &14.6 & 0.15 & 4.83 & \text{Max glow}
    
\end{array}
\end{equation*}

\section{Conclusions} It is uncertain just from looking at
the data whether LEDs are either voltage- or current-driven
devices. This is because both the voltage measured across
the LED and the current going through the LED are increasing
as the LED gets brighter. One might be inclined to think that the more voltage
applied to the LED, the more light is given off, but this idea
fails to acknowledge Ohm's Law: voltage and current are directly
proportional, so as voltage increases, current increases as well,
assuming that the resistance of the resistor stays constant.

In one sentence, if an LED is either current- or voltage-
driven, that means that as either current or voltage
increases, the LED gives off more light.

\(V_f\) (forward voltage) according to the datasheet falls into
the range of \SI{1.8}{V}--\SI{2.2}{V}, and the first sign of
light we observed in the lab was at \SI{1.75}{V}, so it's
pretty close to the expected result. According to the datasheet,
the maximum current that is safe for the LED is \SI{30}{mA}.
We found that at around \SI{6}{mA} the LED is bright enough
to be an indication light, therefore the usable brightness
is achieved at \SI{6}{mA}.

Lastly, it is never acceptable to use an LED without having
some resistance to limit the current. When the voltage applied
reaches the forward voltage, the resistance of the diode falls
to zero, and the current increases hyperbolically. This is
dangerous for the engineer messing with the current, as well as
for the equipment used to generate it.

\end{document}

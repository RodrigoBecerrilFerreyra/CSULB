\documentclass{article}

\usepackage{ragged2e}
\usepackage{graphicx}
\usepackage{amsmath}
\usepackage[binary-units]{siunitx}

\renewcommand{\c}[1]{\texttt{#1}}
%\DeclareBinaryPrefix{\kibi}{Ki}{10}

\begin{document}

\begin{flushright}
    \noindent
    Rodrigo Becerril Ferreyra\\
    CECS 301 Section 03\\
    Lab 5\\
    Due 14 May 2020
\end{flushright}

\section{Introduction} In this lab, we were tasked with creating
a state machine that would control various logic
(supplemental logic)
that would
be used to test and exercise all addresses of a
\(\SI{32}{\kibi b} \times \SI{16}{b} \) memory. Specifically,
we were only tasked with creating the state machine, as the
supplemental logic was given, and the memory was obtained by
using the Vivado IP Catalog library included in most versions
of Vivado (I am using Vivado version 2016.2). The state machine
runs a total of four times, and
fills each address of the memory with
\begin{enumerate}
    \item the address number.
    \item the bitwise NOT of the address number.
    \item hex value \c{0x5555}.
    \item hex value \c{0xaaaa}.
\end{enumerate}

The contents of the address is checked after each pass.
If an error occurred in the process of loading the addresses,
then an error bit is raised. This can help with debugging the
memory by tracing which register was being loaded at the time
the error bit was raised.

\section{State Machine} The state machine was the main part of
the lab, and the piece of the circuit that performed the most
work. It consists of eight states, five inputs (including
\c{clk} and \c{rst}), and four outputs. The basic function
of the state machine is documented by a state transition
diagram. The state machine was programmed using a
modified Moore model, which means that the outputs are not
dependant on the inputs.

\end{document}

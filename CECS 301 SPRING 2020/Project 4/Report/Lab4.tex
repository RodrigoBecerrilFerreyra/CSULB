\documentclass{article}

\usepackage{ragged2e}
\usepackage{graphicx}
\usepackage{amsmath}
\usepackage{siunitx}

%\renewcommand{\c}[1]{\texttt{#1}}

\begin{document}

\begin{flushright}
    \noindent
    Rodrigo Becerril Ferreyra\\
    CECS 301 Section 03\\
    Lab 4\\
    Due 23 April 2020
\end{flushright}

\section{Introduction} In this lab, we were tasked with
modeling an arithmetic logic unit (ALU) and
a register bank or register file. The ALU
is a combinational block which is tasked
with performing various mathematical operations,
such as adding or
subtracting two numbers. The ALU is capable of performing
both unary and binary operations, and the
operation which is selected is
chosen using an opcode.

The register bank is a sequential block which contains many
ragisters. In this lab, our register bank holds eight
\(8\)-bit registers, which is \(64\) bits of total storage. The register
bank only has the ability to write to one register at a time,
so both the data to be written and the address of the register
must be present. However, two registers can be read asynchronously
at a time.

In top-level terms, the design has five inputs---one \(16\)-bit
input and four single bit inputs---and four outputs---
three single bit outputs and one \(32\)-bit output.

\section{Arithmetic Logic Unit} The ALU is similar to a
multiplexer in which it has a control path input that controls
what happens to the datapath inputs to produce an output;
this control input is called the opcode.
In this lab, the ALU can perform \(8\) operations; therefore
the opcode must be \(\lceil\log_2(8)\rceil = 3\) bits wide.
The operations that the ALU can perform on the operands
\texttt{A} and \texttt{B} are denoted in the table below.

\begin{figure}[h]
    \centering
    \begin{tabular}{| c | c |}
        \hline
        Opcode & \(\text{\texttt{Y}} = \text{Output}\)\\ \hline
        0 & \(A + B\) \\ \hline
        1 & \(A - B\) \\ \hline
        2 & \(A << 1\) \\ \hline
        3 & \(\{A[0], A[7:1]\}\) \\ \hline
        4 & \(A \text{ AND } B\) \\ \hline
        5 & \(A \text{ OR } B\) \\ \hline
        6 & \(A\text{ XOR } B\) \\ \hline
        7 & \(\sim A\) \\ \hline

    \end{tabular}
    \caption{ALU Operand Guide}
    \label{fig1}
\end{figure}

The ALU also has two other outputs, both of which are one bit
wide: the \texttt{zero} output is active when \texttt{Y} is
equal to zero, and the \texttt{carry} output is active when
the unsigned addition of \texttt{A} and \texttt{B} results
in an overflow. The two operand inputs of the ALU come from
the register bank, and the opcode input comes from the
top-level \(16\)-bit output.
All outputs of the ALU are sent to the top-level module outputs,
and \texttt{Y} is sent back to the register file to be written.

\section{Register Bank} The register bank is a collection of
registers. Because it is a sequential block, a reset is needed
to bring all eight registers to a known state; the positive edge
of the reset signal sets all registers to zero. The register bank
has a write enable input labeled \texttt{write}: if \texttt{write}
is low, then no register will be written on the active edge of
the clock; if \texttt{write} is high, then on the active edge
of the clock, the register whose address is the value of the
input \texttt{wAdrs} will copy the data on the input
\texttt{wData}. Since there are \(8\) possible registers
to write to, the input \texttt{wAdrs} is \(\lceil\log_2(8)\rceil = 3\)
bits wide; and since each register is \(8\) bits wide,
the input \texttt{wData} is also \(8\) bits wide.

Reading from the register file is handled asynchronously:
the two three-bit wide inputs \texttt{rAdrsA} and \texttt{rAdrsB}
point to the addresses of the registers whose values are outputted
on the two eight-bit wide outputs \texttt{rDataA} and \texttt{rDataB}.
These outputs then feed into both the operand inputs of the
ALU and part of the \(32\)-bit top-level module output.

\end{document}

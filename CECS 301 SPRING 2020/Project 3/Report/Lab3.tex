\documentclass{article}

\usepackage{ragged2e}
\usepackage{graphicx}
\usepackage{amsmath}
\usepackage{siunitx}

%\renewcommand{\c}[1]{\texttt{#1}}

\begin{document}

\begin{flushright}
    \noindent
    Rodrigo Becerril Ferreyra\\
    CECS 301 Section 03\\
    Lab 3\\
    Due 12 March 2020
\end{flushright}

\section{Introduction} In this lab, we were tasked with creating
a three-part circuit: the first part is a circuit that eliminates
the mechanical chatter inherent to buttons and switches; the
second part is a circuit that detects the positive edge of a
signal; and the third part is a counter that feeds into the
seven-segment display that we programmed last lab. The final
behavior of this design is the following: the user pushes a
button and the count on the seven-segment display increments
by one.

\section{Debounce Circuit} To achieve the desired effect of
eliminating the mechanical chatter of a signal, the design
called for shifting in the value of the signal in question
every millisecond (timed using a clock counter/pulse generator)
into a register that is ten bits wide. This will ensure that
the signal is stable for ten milliseconds. Every D-flip-flop
inside this register is then passed through a bitwise AND gate
which is then the output of the positive-edge detector.

\section{Positive Edge Detector} The positive-edge detector
circuit (PED circuit) is used to output a one clock-wide pulse
whenever the input signal experiences a rising edge. The
input to the PED circuit is the signal in question. The circuit
consists of two D-flip-flops, with the signal connected to the
input of the first flop, and the output of the first flop
connected to the input of the second flop. The PED circuit output
is then calculated as follows:
\begin{equation*}
    \text{ped\_out} = Q_1 \cdot \overline{Q_2}
\end{equation*}
This ensures that \texttt{ped\_out} is only one clock wide
and fires when the input is high.

\section{Counter and Seven-Segment Display} The final part of
this lab was to connect the output of the PED circuit to a
counter. The counter is 32 bits wide (to match with the input
of the seven-segment display). It is incremented once and only
once every time the button is pushed thanks to the PED circuit.
The output of the counter is given to the input of the
seven-segment display.

\end{document}

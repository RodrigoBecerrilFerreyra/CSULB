\documentclass{article}

\usepackage{ragged2e}
\usepackage{graphicx}
\usepackage{amsmath}
\usepackage{siunitx}

\renewcommand{\c}[1]{\texttt{#1}}

\begin{document}

\begin{flushright}
    \noindent
    Rodrigo Becerril Ferreyra\\
    CECS 301 Section 03\\
    Lab 1\\
    Due 06 February 2020
\end{flushright}

\section{Introduction} In this lab, we used the HDL Verilog
to simulate strictly combinatorial logic. Combinatorial logic
refers to logic whose output is only dependant on its inputs.
The opposite type of logic is called sequential logic, which
also depends on the output's previous state.
We started by capturing
the desired behavior in a truth table, reducing using K-Maps,
and generating equations. These equations are used in the base-
level full adder. After testing, we can then connect multiple
full adders in order to make multiple-bit full adders, or
vector full adders. In this lab, we made a four-bit adder and an
eight-bit adder using the full adder.

\section{Process} In this lab, the desired logic equations were
given to us. These equations are well known and were derived
in the class CECS 201. They can be found by documenting
the desired action on a truth table, converting to sum-of-
minterms, and reducing using boolean algebra. They are
\c{y = a XOR b XOR ci} and \c{co = a*b + a*ci + b*ci},
where the sum of the three input bits \c{a}, \c{b}, and \c{ci}
is the two output bits \c{\{co,y\}}.

These equations representing combinatorial logic were put into
a module named \c{FA} for ``Full Adder.'' This was then tested
for correctness using a test fixture file provided by the
instructor; its code simply chose three random values for the
three input bits and checked whether or not they added to the
two output bits. Once this was verified, the module was ready
to be instantiated.

Instantiation of a module inside a different module is the
process of using the first module's function in order to save
text space or simplify a process. In this case, four \c{FA}
modules were instantiated inside a different module named \c{FA4}.
This allows two four-bit numbers to be added with a one-bit
carry-in signal, instead of having every input be only one bit.
Three extra wires were required in order to hold the carry-out
bit of a less significant instantiation of the \c{FA} module
and deposit it into the carry-in input of a more significant
instantiation. Again, this was tested using test fixtures
provided by the instructor, and it works the same way.
Once it was verified that the \c{FA4} module was functioning
properly, it was time to move on to the final step.

Lastly, it was required to instantiate two \c{FA4} modules
into one large eight-bit adder, called \c{FA8}. The \c{FA8}
module allows two eight-bit numbers to be added together.
Because eight bits equals one byte, this allows the user to
add two whole bytes of data: one byte can represent numbers
from \(0_{10}\) to \(255_{10}\) unsigned, or \(-128_{10}\) to
\(127_{10}\) signed. This makes \c{FA8} a very powerful module.
The way this was achieved is by simply instantiating two
\c{FA4} modules; this required an extra wire to hold the
carry-out output of the first module and feed it into the carry-
in input of the second module. Once this module was tested
using the test fixture provided by the instructor, the
lab was completed.

\section{Conclusion} In this lab, I reviewed how to model
combinatorial logic in Vivado, and how to instantiate modules
to make other modules. I learned this last semester in CECS 201,
but this lab was very helpful in reviewing this very important
topic. This process of instantiation and building a working
foundation in order to build higher-level modules is an
important one that will most likely be critical in future labs.

\end{document}

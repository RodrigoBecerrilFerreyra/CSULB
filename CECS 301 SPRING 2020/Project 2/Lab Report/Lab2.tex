\documentclass{article}

\usepackage{ragged2e}
\usepackage{graphicx}
\usepackage{amsmath}
\usepackage{siunitx}

%\renewcommand{\c}[1]{\texttt{#1}}

\begin{document}

\begin{flushright}
    \noindent
    Rodrigo Becerril Ferreyra\\
    CECS 301 Section 03\\
    Lab 2\\
    Due 25 February 2020
\end{flushright}

\section{Introduction} In this lab, we used the language Verilog
to be able to use the seven-segment display on the
Nexys A7 FPGA board. A seven-segment display is a tool
to display values internal to the FPGA. It can be used to
display the contents of a register, but in this lab, we are
using it to display the value presented on the switches.
We can later reuse the code in order to display values on
an internal register to aid with debugging a module.

Displaying a value on the seven-segment display is more complex
than it sounds, because many different factors come into play.
First, it is necessary to choose which four switches are going
to be displayed. This is because only one set of four bits
can be displayed at a time. This is achieved using
a multiplexer, which will
take four of the bits that are
presented via the switches and feed it into a decoder which
will decode it into eight bits; these decoded eight bits
will go into the cathode inputs of the seven-segment display.
The LEDs only have one set of eight cathode inputs, so we need
a multiplexer in order to select which four switches go to
which eight inputs. Since there are only eight inputs,
there
is an enable switch to display which LED the eight inputs go
into. This is controlled by a rotating shift register, which
only allows one active signal to pass through it. Lastly,
in order to see the changes happening, it is necessary to
slow the frequency of the FPGA's internal clock down. Since
this cannot be done directly, it is necessary to create a one-
clock-wide pulse every \(x\) amount of clocks. In this lab,
we are creating a pulse every \num{5000} clock cycles, and 
assuming that the FPGA's clock is set to \SI{100}{\mega\hertz},
our pulse will be \SI{20}{\kilo\hertz}.

\section{Process} In total, there are five different modules
to make: a 4-bit 8:1 multiplexer, a custom 4:8 decoder, a
pulse generator, a 3-bit counter, and a shift register. At
first, I made all five modules separately and instantiated
them in a separate top-level module, but then cut and pasted
the code from the five modules into one, for easy re-use.

\subsection{Multiplexer, Decoder, and Counter} The multiplexer
was simple to construct:
it is a purely combinatorial circuit, and all it takes is one
\texttt{always @(*)} block in order to create one. Because
the multiplexer is 32 bits wide total, and the top-level
input is only 16 bits, it was necessary to copy the same
inputs into the top 16 bits of the mux.

The decoder is another purely combinatorial circuit. It takes
a 4-bit input from the mux and turns it into an 8-bit
output. The output logic is in accordance to standard
seven-segment display procedures. The output of this
decoder is connected to the \texttt{cathode} output of the
top-level module.

The select signal of the mux is controlled by a counter; this
counter counts from \num{0} to \num{7} using three bits; these
bits are fed into the mux. The counter increments synchronously
with the 8-bit shift register, because the counter's ``increment''
input and the register's ``shift'' input are tied to the pusle
generator's ``pulse'' output. This allows the the mux's
\(n\text{-th}\) input to coincide with having a zero in the
register's \(n\text{-th}\) position.

\subsection{Pulse Generator} The pulse generator is an integral
part of this design, as it allows the user to see the numbers.
If no pulse generator were present, and the ``increment'' and
``shift'' inputs of the counter and shift register respectively
were directly connected to the clock, the LEDs of the
seven-segment display would turn on and off too quickly, and no
number would show.

The pulse generator works by having an
internal counter which counts clock cycles. When a fixed number
of clock cycles \(x\) is reached, the pulse generator outputs a
pulse and the counter gets set to zero. \(x\) is calculated by
dividing the internal clock frequency of the FPGA by the desired
clock frequency. Originally, my desired clock frequency was
\SI{2}{\kilo\hertz}, but I experienced annoying flashing of the
LEDs, so I increased it by a factor of ten. Therefore, my new
desired frequency is \SI{20}{\kilo\hertz}, and the number of
clocks \(x = 5000\).

The output of the pulse generator is hooked up to the
``increment'' and
``shift'' inputs of the counter and shift register, respectively.

\subsection{Shift Register} The last piece of the puzzle is
the shift register. Specifically, it is a rotating left shift
register; this means that the register only shifts left, and
whatever bit is at the MSB position will be inserted into the
LSB position. This behavior can be achieved in a variety of
ways, such as using an \texttt{if} statement, but I decided
to use the concatenation operator. The output of this register
is connected to the \texttt{anode} output of the module.

\section{Conclusion} In this lab, I learned how to efficiently
model sequential logic and separate it from combinatorial logic. I
also learned how to
write a top-level module in two ways---by making all the modules
separately and instantiating them in a separate module, or by 
writing all the code in a single module---and the pros and
cons of each format. Overall, this lab will help me with future
labs, as I gained much experience.

\end{document}

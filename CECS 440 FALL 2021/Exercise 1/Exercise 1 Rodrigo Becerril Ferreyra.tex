\documentclass{article}

\usepackage{graphicx} % for images
\usepackage{amsmath} % for math
\usepackage{amssymb} % for \mathbb
\usepackage{siunitx} % for \SI, \num
\usepackage{hyperref} % for \url{}

% This stuff is for figures
\usepackage{float}
\DeclareGraphicsExtensions{.pdf, .png, .jpg}

% coloring of links for PDF format
\hypersetup{
    colorlinks=true,
    urlcolor=blue,
    linkcolor=black
}

% \c command redefinition (for monospaced font)
\renewcommand{\c}[1]{\texttt{#1}}
% \today command re-definition
%https://tex.stackexchange.com/questions/112932/today-month-as-text
\renewcommand{\today}{\ifnum\number\day<10 0\fi \number\day \space%
\ifcase \month \or January\or February\or March\or April\or May%
\or June\or July\or August\or September\or October\or November\or December\fi\space%
\number \year} 

\begin{document}

\noindent
Rodrigo Becerril Ferreyra\\
CECS 440 Section 02\\
Exercise 1\\
\today

\addcontentsline{toc}{section}{Problem 1.1}
\section*{Problem 1.1}
Three different types of computers include the following:
\begin{itemize}
    \item Personal computers (PCs)
    \item Server computers
    \item Embedded computers
\end{itemize}

\addcontentsline{toc}{section}{Problem 1.3}
\section*{Problem 1.3}
A computer program written in C goes through the preprocessor
which prepares the program for compilation, the compiler
which turns the high-level code into assembly, the assembler
which translates the compiler's output into machine (binary)
code, and finally the linker, where external libraries can be
added to the final program.

\addcontentsline{toc}{section}{Problem 1.5}
\section*{Problem 1.5}
\begin{enumerate}
    \item P2 has the highest performance in instructions
    per second, at \num{2.5e9}:
    \begin{equation*}
        \frac{\num{2.5e9}\text{ cyc}}{1\text{ s}}\times
        \frac{1\text{ inst}}{1\text{ cyc}}
        = \frac{\num{2.5e9}\text{ inst}}{1\text{ s}}
    \end{equation*}
    P1 and P3 have an instructions per second rating of \num{2e9}
    and \num{1.818e9}, respectively.
    \item \begin{itemize}
        \begin{tabular}{|l|l|l|}
            \hline
            In 10 seconds: & Number of Cycles & Number of instructions \\ \hline
            P1 & \num{30e9} & \num{20e9} \\ \hline
            P2 & \num{25e9} & \num{25e9} \\ \hline
            P3 & \num{40e9} & \num{18.18e9} \\ \hline
            \end{tabular}
    \end{itemize}
    \item We should have a clockrate that is
    \(1.2 \times 1.3 = 1.56x\) the rate of the original clock;
    that is, the clock should be 56\% higher.
\end{enumerate}

\addcontentsline{toc}{section}{Problem 1.7}
\section*{Problem 1.7}
There are \num{1e5} class A instructions,
\num{2e5} class B instructions, \num{5e5} class C instructions,
and \num{2e5} class D instructions.
\begin{enumerate}
    \item The number of cycles P1 will go through to complete
    the program is
    \(1(\num{1e5}) + 2(\num{2e5}) + 3(\num{5e5}) + 3(\num{2e5}) = \num{2.6e6}\)
    cycles,
    and the number of cycles P2 will go through to
    complete the program is
    \(2(\num{1e5}) + 2(\num{2e5}) + 2(\num{5e5}) + 2(\num{2e5}) = \num{2.0e6}\)
    cycles.
    \item P1 will take \(\frac{\num{2.6e6}\text{ cyc}}{\num{2.5e9}\text{ cyc/s}} = \SI{1.04}{\milli\second}\)
    to complete the program, while P2 will take
    \(\frac{\num{2.0e6}\text{ cyc}}{\num{3e9}\text{ cyc/s}} = \SI{667}{\micro\second}\)
    to complete the same program. P2 is faster.
\end{enumerate}

\addcontentsline{toc}{section}{Problem 1.9}
\section*{Problem 1.9}
\begin{gather*}
    P = \frac{1}{2} C V^2 f\\
    C = \frac{2P}{V^2f}
\end{gather*} where \(P\) is power, \(C\) is capacitive load,
\(V\) is voltage, and \(f\) is frequency.
\begin{enumerate}
    \item The Pentium 4 Prescott processor has an average
    capacitive load of
    \begin{equation*}
        \frac{2(\SI{100}{\watt})}{(\SI{1.25}{\volt})^2\SI{3.6}{\giga\hertz}}
        = \SI{35.6}{\nano\farad}.
    \end{equation*}
    The Core i5 Ivy Bridge processor has an average capacitive load of
    \begin{equation*}
        \frac{2(\SI{70}{\watt})}{(\SI{0.9}{\volt})^2 \SI{3.4}{\giga\hertz}}
        = \SI{50.8}{\nano\farad}.
    \end{equation*}
    \item The Pentium 4 Prescott processor idle power dissipation ratio
    is
    \begin{equation*}
        \SI{10}{\watt}/(\SI{10}{\watt} + \SI{90}{\watt}) = 10\%.
    \end{equation*}
    The Core i5 Ivy Bridge processor idle power dissipation ratio
    is
    \begin{equation*}
        \SI{30}{\watt}/(\SI{30}{\watt} + \SI{40}{\watt}) \approx 42.9\%.
    \end{equation*}
    \item The voltage should also be reduced by 10\% to keep the
    same leakage current.
\end{enumerate}

\addcontentsline{toc}{section}{Problem 1.13}
\section*{Problem 1.13}
\begin{enumerate}
    \item The equation for CPU time is
    \begin{equation*}
        t_\text{CPU} = \text{instcount} \times \text{CPI} \times \frac1f.
    \end{equation*}
    Processor P1 has a CPU time of \((\num{5e9})\times (0.9) \times \frac{1}{\SI{4}{\giga\hertz}} = \SI{1.125}{\second}\).
    Processor P2 has a CPU time of \((\num{1e9})\times (0.75) \times \frac{1}{\SI{3}{\giga\hertz}} = \SI{0.25}{\second}\).
    Even though P1 has a higher clock rate, P2 beats P1 in total time spent.
    \item P1 will take 0.\SI{0.225}{\second} to process \num{1e9}
    instructions. The number of instructions \(\iota\) that P2
    can process in the same amount of time is:
    \begin{gather*}
        t_\text{CPU} = \text{instcount} \times \text{CPI} \times \frac1f\\
        \SI{0.225}{\second} = \iota \times \frac{0.75}{\SI{3}{\giga\hertz}}\\
        \iota = \num{9e8}
    \end{gather*}
    P2 can process \num{9e8} instructions in the same time
    that it takes P1 to process \num{10e8}, which is less.
    \item P1 has a MIPS rating of \(\frac{\SI{4}{\giga\hertz}}{\num{0.9e6}} = \SI{4444}{\per\second}\),
    while P2 has a MIPS rating of \(\frac{\SI{3}{\giga\hertz}}{\num{0.75e6}} = \SI{4000}{\per\second}\).
    P1 has a higher MIPS rating.
    \item P1's MFLOPS rating is \(\frac{0.4(\num{5e9})}{\SI{1.125e6}{\second}} = \SI{1778}{\per\second}\),
    while P2 MFLOPS rating is \( \frac{0.4(\num{1e9})}{\SI{0.25e6}{\second}} = \SI{1600}{\per\second}\).
    P1 has a higher MFLOPS rating.
\end{enumerate}

\addcontentsline{toc}{section}{Problem 1.15}
\section*{Problem 1.15}
Amdahl's Law states that
\begin{equation*}
    t = o/n + u
\end{equation*} where \(t\) is the execution time after
improvement, \(o\) is the amount of time of the thing
you're improving (before improvement), \(n\) is the amount
of time you're improving it by, and \(u\) is the time of
the unaffected component of the total time.
The amount of time the processor takes to run all
instructions at current specs as given in the problem is
given in the table below.

\begin{table}[H]
    \centering
    \begin{tabular}{l|l}
        Type of Instruction & Time (s) \\ \hline
        FP & 0.025 \\
        INT & 0.055 \\
        L/S & 0.16 \\
        branch & 0.016 \\
        Total & 0.256
    \end{tabular}
\end{table}

\begin{enumerate}
    \item Half the time it takes for the program to run by improving FP operations:
    \begin{gather*}
        \SI{0.128}{\second} = \frac{\SI{0.025}{\second}}{n} + \SI{0.231}{\second}\\
        -\SI{0.103}{\second} = \frac{\SI{0.025}{\second}}{n}\\
        n = -0.243
    \end{gather*}
    It is impossible to half the runtime because even if the
    FP operations were to disappear, the program would still
    run for \SI{0.231}{\second}.
    \item Half the time it takes for the program to run by improving L/S operations:
    \begin{gather*}
        \SI{0.128}{\second} = \frac{\SI{0.16}{\second}}{n} + \SI{0.096}{\second}\\
        \SI{0.032}{\second} = \frac{\SI{0.16}{\second}}{n}\\
        n = 5
    \end{gather*}
    The CPI of the L/S instruction must be improved by 5x,
    meaning the CPI should be at 0.2 CPI.
    \item The old, unimproved run time of the program is \SI{0.256}{\second},
    and the new improved time obtained by improving FP and INT by
    40\%, and L/S and branch by 30\%, is
    \begin{equation*}
        \left(0.025\cdot0.6\right)+\left(0.055\cdot0.6\right)+\left(0.16\cdot0.7\right)+\left(0.016\cdot0.7\right)=\SI{0.1712}{\second}
    \end{equation*}
    This means that there is an improvement in time of
    \(\frac{0.256}{0.1712}\approx 1.49\).
\end{enumerate}

\end{document}
